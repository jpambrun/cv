%TODO: add details for committee membership. 

\documentclass[print]{friggeri-cv}
\usepackage{enumitem}
\addbibresource{bibliography.bib}
\tolerance=1
\emergencystretch=\maxdimen
\hyphenpenalty=10000
\hbadness=10000
\renewcommand\labelitemi{$\cdot$}
\setlist[itemize]{leftmargin=*}

\begin{document}

\header{jean-françois}{pambrun}{researcher | engineer | software developer}

\newcommand{\eletr}{\href{http://www.etsmtl.ca/Programmes-Etudes/1er-cycle/Fiche-de-cours?Sigle=ELE747}{ele{\footnotesize747}}}
\newcommand{\eleenv}{\href{http://www.etsmtl.ca/Programmes-Etudes/1er-cycle/Fiche-de-cours?Sigle=ELE116}{ele{\footnotesize116}}}
\newcommand{\infmat}{\href{http://www.polymtl.ca/etudes/cours/details.php?sigle=INF1005a}{inf{\footnotesize1005}a}}
\newcommand{\gtssys}{\href{http://www.etsmtl.ca/Futurs-etudiants/Cycles-sup/Fiche-de-cours?Sigle=GTS840}{gts{\footnotesize840}}}

% In the aside, each new line forces a line break
\begin{aside}
  \section{about}
  {\small(514) 222-2085}
  {\small6783} chambord
  montréal
  canada
  ~
  \href{mailto:jf.pambrun@gmail.com}{jf.pambrun@gmail.com}
  \href{https://ca.linkedin.com/in/jpambrun}{linkedin://jpambrun}
  \href{http://github.com/jpambrun}{github://jpambrun}
  % \href{http://jfpb.net}{http://jfpb.net}
  \section{languages}
  french/english
  \section{programming}
  java, javascript
  matlab, python
  linux, git, c, c{\tiny$^{++}$}
  latex, tensorflow
  \section{standards}
  dicom, ihe, hl{\small7}
  jpeg {\small2000}, jpip
  \section{interests}
  medical imaging\\image compression\\deep learning\\rendering
  ~\vspace{2cm}
  {\footnotesize\textdagger~ not a member of the OIQ\\\textdaggerdbl~ fast-track doctoral admission without obtaining a master's degree}
\end{aside}

% \begin{cvfootnote}
%   \textdagger~ not a member of the OIQ\\\textdaggerdbl~ fast-track doctoral admission without obtaining a master's degree
% \end{cvfootnote}


\section{education}
\begin{entrylist}
  \entry {2011-2016} 
  {Ph.D. in electrical engineering} 
  {École de technologie supérieure {\scriptsize (ÉTS)}}
  {improving medical image compression and transmission
  \begin{itemize}
    \item develop a novel image quality metric adapted to diagnostic imaging
    \item propose a novel jpeg 2000 bit allocation mechanism
    \item improve streaming for large image series (ct, mr, tomo, etc.)
    \item skills: research, compression, streaming, matlab, java, c++, itk/vtk
  \end{itemize}}

  \entry
  {2009-2010}
  {M.Sc. in electrical engineering (incomplete\textsuperscript{\tiny\textdaggerdbl})}
  {École de technologie supérieure {\scriptsize (ÉTS)}}
  {evaluation of the diagnostic quality of lossy compressed medical images
  \begin{itemize}
    \item research medical image quality assessment and diagnostic losslessness
    \item quantify the ct acquisition parameters that affect compressibility
    \item skills: research, compression, matlab, java, c++, itk/vtk, cuda
  \end{itemize}}

  \entry {2005-2009} {B.ing. in electrical engineering}
  {École de technologie supérieure {\scriptsize (ÉTS)}}
  {\emph{information technologies and telecommunication}}

\end{entrylist}

\vspace{5mm}

\section{experience}

\begin{entrylist}
  \entry {2016 (dec-)}
  {software developer} {nucleus.io}
  {cloud-based diagnostic workstation with client-side {\small MPR}
  \begin{itemize}
    \item design and implement a client-side multi-planar reconstruction renderer
    \item design and implement a 3d compression algorithm for fast streaming
    \item design and implement annotation tools such as length, Cobb angle, etc 
    \item write highly optimized code using the latest web technologie
    \item ensure support for ct, mr, pet, ultrasound, large tomosynthesis, etc
    \item skills: javascript, nosql, mongodb, python, nodejs, webgl, rendering
  \end{itemize}}

  \entry {2016 (jun-dec)} 
  {postdoctoral fellow}
  {CHUM research centre}
  {improve image-guided prostate cancer brachytherapy treatments
  \begin{itemize}
    \item implement MR-Ultrasound fusion and segmentation using machine learning
    \item study the impact of dual energy ct dect) on current fusion algorithms
    \item evaluate an experimental non-rigid mr-us fusion workflow in the operating room
    \item skills: python, tensorflow, machine learning, registration, segmentation
  \end{itemize}}


  \entry {2016 (sep-dec)}
  {lecturer}
  {École polytechnique de montréal} 
  {procedural programming (\infmat)
  \begin{itemize}
    \item teach introductory procedural programming using matlab
    \item skills: communication, leadership
  \end{itemize}}

\end{entrylist}

\clearpage
% \header{jean-françois}{pambrun}{researcher | engineer | software developer}
\section{experience (cont.)}
\begin{entrylist}
  \entry {2009-2016} 
  {lecturer}
  {École de technologie supérieure {\scriptsize (ÉTS)}}
  {software development environment (\eleenv)
  \begin{itemize}
    \item teach object oriented programming with java
    \item discuss development tools such as ide, debuggers and git 
  \end{itemize}
  image processing and analysis (\eletr)
    \begin{itemize}
    \item teach image processing techniques using matlab
    \item discuss topics such as fft, edge detectors and compression  
  \end{itemize}
  }

  \entry {2007-2014}
  {teaching assistants}
  {École de technologie supérieure {\scriptsize (ÉTS)}}
  {\emph{healthcare distributed systems (\gtssys)
  \begin{itemize}
    \item supervise and grade graduate student lab assignments 
    \item discuss topics such as hl7, ihe and dicom
  \end{itemize}
  software development environment (\eleenv)\\
  image processing and analysis (\eletr)}}
  
  \entry {2009 (jan-sep)}
  {software developer} {\scriptsize{CAE} inc.}
  {head-up display simulation for military flight simulators
  \begin{itemize}
    \item implement a c/c++ modules to stimulate and simulate avionic systems
    \item implement an opengl solution to simulate a hud
    \item work with clients to fix issues and achieve acceptance
    \item skills: c, c++, pascal, opengl, simulation
  \end{itemize}
  }

  \entry {2006-2008}
  {research assistant}
  {École de technologie supérieure {\scriptsize (ÉTS)}}
  {implementation of a standard compliance validation tool for hl{\small7}v{\small3}
  \begin{itemize}
    \item implement a hl{\small7}v{\small3} for validation prior to connectathon
    \item provide support for implementors
  \item skills: java, xml, xml schemas, xslt, soap
  \end{itemize}
  implementation and evaluation of a medical image streaming framework
  \begin{itemize}
    \item evaluate jpip for large image stacks and large image streaming
    \item skills: java, jpip, jpeg 2000
  \end{itemize}
  }
\end{entrylist}

\section{awards}
\begin{entrylist}
  \awardentry {2017} {{\small NSERC} postdoctoral fellowship (declined)} {90,000\$} {\vspace{-3mm}}
  \awardentry {2017} {{\small FRQNT} postdoctoral fellowship (declined)} {70,000\$} {\vspace{-3mm}}
  \awardentry {2016} {{\small GRSTB} postdoctoral fellowship} {18,000\$} {\vspace{-3mm}}
  \awardentry {2011} {{\small NSERC} doctoral Alexander-Graham-Bell scholarship} {105,000\$} {\vspace{-3mm}}
  \awardentry {2011} {{\small ÉTS} excellence graduate student scholarship} {60,000\$} {\vspace{-3mm}}
  \awardentry {2011} {{\small FRQNT} doctoral research scholarship (declined)} {60,000\$} {\vspace{-3mm}}
  \awardentry {2009} {{\small FRQNT} master's research scholarship} {30,000\$} {\vspace{-3mm}}
  \awardentry {2006} {{\small NSERC} undergraduate student research award} {4,500\$} {\vspace{-3mm}}
 \end{entrylist}
\vspace{2mm}

\section{committee and board memberships}
\begin{entrylist}
 \entry {2013-2016} {board member of an {\small NPO}} {Association étudiante de l'{\scriptsize ÉTS}} {vice-president of services\\ academic commission student representative \\ disciplinary committee student representative}
\end{entrylist}



\section{publications}
\vspace{-3mm}
\newrefcontext[sorting=chronological]
\nocite{*}
\printbibliography[
  type=article,
  title={articles in peer-reviewed journals},
  heading=subbibliography]
\printbibliography[
  type=inproceedings,
  title={international peer-reviewed conferences/proceedings},
  heading=subbibliography]

\end{document}
