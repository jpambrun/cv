%!TEX TS-program = xelatex
%\def\isenglish{1}
%\def\isprint{1}

\ifdefined\isprint
  \documentclass[print]{friggeri-cv}
\else
  \documentclass[]{friggeri-cv}
\fi
\addbibresource{bibliography.bib}
%TODO 3 articles
\tolerance=1
\emergencystretch=\maxdimen
\hyphenpenalty=10000
\hbadness=10000
\begin{document}

\engfr{
\header{jean-françois}{pambrun}{researcher | engineer | software developer}
}{
\header{jean-françois}{pambrun}{chercheur | ingénieur | développeur}
}

\newcommand{\eletr}{\href{http://www.etsmtl.ca/Programmes-Etudes/1er-cycle/Fiche-de-cours?Sigle=ELE747}{ele{\footnotesize747}}}
\newcommand{\eleenv}{\href{http://www.etsmtl.ca/Programmes-Etudes/1er-cycle/Fiche-de-cours?Sigle=ELE116}{ele{\footnotesize116}}}
\newcommand{\gtssys}{\href{http://www.etsmtl.ca/Futurs-etudiants/Cycles-sup/Fiche-de-cours?Sigle=GTS840}{gts{\footnotesize840}}}

% In the aside, each new line forces a line break
\begin{aside}
%    \ifdefined\isenglish\section{about}\else\section{à propos}\fi
  \section{\engfr{about}{à propos}}
  {\small(514) 222-2085}
  {\small8150} fabre
  montréal
  canada
  ~
  \href{mailto:jf.pambrun@gmail.com}{jf.pambrun@gmail.com}
  \href{https://ca.linkedin.com/in/jpambrun}{linkedin://jpambrun}
  \href{http://github.com/jpambrun}{github://jpambrun}
  \href{http://jfpb.net}{http://jfpb.net}
  \section{\engfr{languages}{langues}}
  \engfr{bilingual french/english}{bilingue français/anglais}
  \section{\engfr{programming}{logiciels}}
  java, javascript
  matlab, c, c{\tiny$^{++}$}
  linux, xml, git
  html{\small5}, latex
  \section{\engfr{standards}{standards}}
  dicom, ihe, hl{\small7}
  jpeg {\small2000}, jpip
\end{aside}

\begin{cvfootnote}
  \engfr{\textdagger~ not a member of the OIQ\\\textdaggerdbl~ fast-track doctoral admission without obtaining a master's degree }{\textdagger~ non membre de l'OIQ\\\textdaggerdbl~ passage direct au doctorat sans obtenir le grade de maîtrise}
\end{cvfootnote}
% \section{interests}

% complex networks, social networks, community detection, community structure,
% overlapping communities, information diffusion, viral marketing, social
% inference, recommendation, data mining

\engfr{\section{education}}{\section{éducation}}

\begin{entrylist}
  \engfr
  {\entry {since 2011} {Ph.D. in electrical engineering (in progress)} {École de technologie supérieure {\scriptsize (ÉTS)}} {\emph{improving medical image compression and transmission}}}
  {\entry {depuis 2011} {Ph.D. en génie électrique (en cours)} {École de technologie supérieure {\scriptsize (ÉTS)}} {\emph{optimisation de la compression et de la transmission d'images médicales}}}

  \engfr
  {\entry {2009-2010} {M.Sc. in electrical engineering (incomplete\textsuperscript{\tiny\textdaggerdbl})} {École de technologie supérieure {\scriptsize (ÉTS)}} {\emph{evaluation of the diagnostic quality of lossy compressed medical images}}}
  {\entry {2009-2010} {M.Sc. en génie électrique (incomplet\textsuperscript{\tiny\textdaggerdbl})} {École de technologie supérieure {\scriptsize (ÉTS)}} {\emph{évaluation des pertes liées à la compression d'images médicales}}}

  \engfr
  {\entry {2005-2009} {B.ing. in electrical engineering} {École de technologie supérieure {\scriptsize (ÉTS)}} {\emph{information technologies and telecommunication}}}
  {\entry {2005-2009} {B.ing. en génie électrique} {École de technologie supérieure {\scriptsize (ÉTS)}} {\emph{concentration technologie de l'information et télécommunications}}}

  \engfr
  {\entry {2001-2005} {Diploma of college studies in electrical engineering} {Cégep de l’Outaouais} {\emph{computerized system technologies}}}
  {\entry {2001-2005} {Diplôme d'études collégiales en génie électrique} {Cégep de l’Outaouais} {\emph{concentration technologie de systèmes ordinés}}}
\end{entrylist}


% \cventry{2008--2009}{Concepteur de matériel didactique}{École de technologie supérieure}{}{}{
% Responsable de l’élaboration des exercices et des outils nécessaires au bon déroulement des laboratoires du cours Systèmes répartis dans le domaine de la santé ((gts\footnotesize840})).
% \begin{itemize}
%   \item Assembler/développer des logiciels de type clients/serveurs suivant les spécifications et standards HL7 et DICOM pour les dossiers patients
%   \item Concevoir, créer et mettre au point les énoncés de laboratoire et les méthodes nécessaires à l’évaluation des étudiants
% \end{itemize}}

\engfr{\section{experience}}{\section{expérience}}

\begin{entrylist}
  \engfr
  {\entry {2009-2014} {lecturer} {École de technologie supérieure {\scriptsize (ÉTS)}} {\emph{software development environment (\eleenv)}\\image processing and analysis (\eletr) }}
  {\entry {2009-2014} {chargé de cours} {École de technologie supérieure {\scriptsize (ÉTS)}} {\emph{environnement de développement de logiciels (\eleenv)}\\analyse et traitement d'images (\eletr)}}

  \engfr
  {\entry {2007-2014} {teaching assistants} {École de technologie supérieure {\scriptsize (ÉTS)}} {\emph{healthcare distributed systems (\gtssys)\\software development environment (\eleenv)\\image processing and analysis (\eletr)}}}
  {\entry {2007-2014} {chargé de laboratoire} {École de technologie supérieure {\scriptsize (ÉTS)}} {\emph{systèmes répartis dans le domaine de la santé (\gtssys)\\environnement de développement de logiciels (\eleenv)\\analyse et traitement d'images (\eletr)}}}

  \engfr
  {\entry {01-09 2009} {software developer} {\scriptsize{CAE} inc.} {\emph{head-up display simulation for military flight simulators}}}
  {\entry {01-09 2009} {développeur logiciel} {\scriptsize{CAE} inc.} {\emph{simulation d'affichage tête haute pour simulateurs de vol militaire}}}

  \engfr
  {\entry {2006-2008} {research assistant} {École de technologie supérieure {\scriptsize (ÉTS)}} {\emph{implementation of a standard compliance validation tool for hl{\small7}v{\small3}\\implementation and evaluation of a medical image streaming platform}}}
  {\entry {2006-2008} {auxiliaire de recherche} {École de technologie supérieure {\scriptsize (ÉTS)}} {\emph{développement d'un outil de validation de conformité au standard hl{\small7}v{\small3}.\\Implémentation et évaluation d'un système de <<streaming>> d'images médicales.}}}
\end{entrylist}

\engfr{\section{awards}}{\section{récompenses}}

\begin{entrylist}
  \engfr
  {\entry {2014} {{\small SIIM} annual meeting best scientific poster} {500\$} {\vspace{-3mm}}}
  {\entry {2014} {meilleur affiche scientifique à la conférence annuelle {\small SIIM}} {500\$} {\vspace{-3mm}}}

  \engfr
  {\entry {2011} {{\small NSERC} doctoral Alexander-Graham-Bell scholarship} {35,000\$/yr} {\vspace{-3mm}}}
  {\entry {2011} {bourse d’étude supérieure Alexander-Graham-Bell du {\small CRSNG}} {35,000\$/an} {\vspace{-3mm}}}

  \engfr
  {\entry {2011} {{\small ÉTS} excellence graduate student scholarship} {20,000\$/yr} {\vspace{-3mm}}}
  {\entry {2011} {bourse d'excellence pour les diplômés de 1er cycle de l'ÉTS} {20,000\$/an} {\vspace{-3mm}}}

  \engfr
  {\entry {2011} {{\small FRQNT} doctoral research scholarship} {20,000\$/yr} {\vspace{-3mm}}}
  {\entry {2011} {bourse de doctorat en recherche du {\small FRQNT}} {20,000\$/an} {\vspace{-3mm}}}

  \engfr
  {\entry {2009} {{\small FRQNT} master's research scholarship} {15,000\$/yr} {\vspace{-3mm}}}
  {\entry {2009} {bourse de maîtrise en recherche du {\small FRQNT}} {15,000\$/an} {\vspace{-3mm}}}

  \engfr
  {\entry {2006} {{\small NSERC} undergraduate student research awards} {4,500\$} {\vspace{-3mm}}}
  {\entry {2006} {bourse de recherche de 1er cycle du {\small CRSNG}} {4,500\$} {\vspace{-3mm}}}
\end{entrylist}

\engfr{\section{volunteer work}}{\section{implication bénévole}}
\begin{entrylist}
  \engfr
  {\entry {since 2013} {{\small NPO} board member} {Association étudiante de l'{\scriptsize ÉTS}} {vice-president of services\\ academic commission student representative \\ disciplinary committee student representative}}
  {\entry {depuis 2013} {administrateur d'un {\small OBNL}} {Association étudiante de l'{\scriptsize ÉTS}} {vice-président aux services\\représentant étudiant à la commission des études\\représentant étudiant au comité de discipline}}
\end{entrylist}
\newpage
% \section{bourses et récompenses}
% \cventry{2011}{\small bourse d’études supérieures Alexander-Graham-Bell du {\small CRSNG}{35,000\$/an}{}{}{
% La prestigieuse bourse d’études supérieures du Canada (BESC) Alexander-Graham-Bell a été créée afin d’assurer un bassin fiable de personnes hautement qualifiées en vue de satisfaire aux besoins de l’économie du savoir du Canada. Les titulaires d’une BESC Alexander-Graham-Bell aideront à renouveler le corps professoral des universités canadiennes et seront les chefs de file de demain en recherche.}
% \cventry{2011}{\small bourse d’excellence pour les diplômés de 1er cycle de l’\textsc{éts}}{20,000\$/an}{}{}{}
% \cventry{2011}{\small bourse de doctorat en recherche du {\small FRQNT}}{20,000\$/an}{}{}{}
% \cventry{2009}{\small bourse de maîtrise en recherche du {\small FRQNT}}{15,000\$/an}{}{}{}
% \cventry{2006}{\small bourse de recherche de 1er cycle du {\small CRSNG}{4,500\$}{}{}{}
% \section{publications}


\engfr{\section{publication}}{\section{publication}}
\vspace{-3mm}
\printbibsection{article}{\ifdefined\isenglish articles in peer-reviewed journals\else articles de journaux évalués par les pairs\fi}
\begin{refsection}
  \nocite{*}
  \printbibliography[sorting=chronological, type=inproceedings, title={\ifdefined\isenglish international peer-reviewed conferences/proceedings \else articles de conférences internationales évalués par les pairs \fi}, notkeyword={france}, heading=subbibliography]
\end{refsection}
% \begin{refsection}
  % \nocite{*}
  % \printbibliography[sorting=chronological, type=inproceedings, title={local peer-reviewed conferences/proceedings}, keyword={france}, heading=subbibliography]
% \end{refsection}
% \printbibsection{misc}{other publications}
%\printbibsection{report}{rapport technique}

\end{document}
