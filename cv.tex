%!TEX TS-program = xelatex
\documentclass[]{friggeri-cv}
\addbibresource{bibliography.bib}
\newif\ifenglish
%\englishtrue % Show English text only
%TODO 3 articles

\begin{document}

\ifenglish
\header{jean-françois}{pambrun}{researcher | engineer | software developer}
\else
\header{jean-françois}{pambrun}{chercheur | ingénieur | développeur}
\fi


% In the aside, each new line forces a line break
\begin{aside}
    \ifenglish\section{about}\else\section{à propos}\fi
    {\small8150} rue fabre
    montréal
    canada
    ~
    \href{mailto:jf.pambrun@gmail.com}{jf.pambrun@gmail.com}
    \href{https://ca.linkedin.com/in/jpambrun}{linkedin://jpambrun}
    \href{http://github.com/jpambrun}{github://jpambrun}
    \ifenglish\section{languages}\else\section{langue}\fi
    \ifenglish bilingual french/english \else bilingue français/anglais \fi
    \ifenglish\section{programming}\else\section{logiciels}\fi
    java, javascript
    matlab, c, c{\tiny$^{++}$}
    linux, xml, git
    html{\small5}, latex
    \ifenglish\section{standards}\else\section{standards}\fi
    dicom, ihe, hl{\small7}
    jpeg {\small2000}, jpip
\end{aside}

% \section{interests}

% complex networks, social networks, community detection, community structure,
% overlapping communities, information diffusion, viral marketing, social
% inference, recommendation, data mining

\ifenglish\section{education}\else\section{éducation}\fi

\begin{entrylist}
  \ifenglish
    \entry {since 2011} {Ph.D. in electrical engineering} {École de technologie supérieure} {\emph{TODO}}
  \else
    \entry {depuis 2011} {Ph.D. en génie électrique (en cours)} {École de technologie supérieure} {\emph{Optimisation de la compression et de la transmission d'images médicales.}}
  \fi

  \ifenglish
    \entry {2009-2010} {M.Sc. in electrical engineering (incomplete)} {École de technologie supérieure} {\emph{TODO}}
  \else
    \entry {2009-2010} {M.Sc. en génie électrique (passage direct)} {École de technologie supérieure} {\emph{Évaluation des pertes liées à la compression d'images médicales.}}
  \fi

  \ifenglish
    \entry {2005-2009} {B.ing. in electrical engineering} {École de technologie supérieure} {\emph{TODO}}
  \else
    \entry {2005-2009} {B.ing. en génie électrique} {École de technologie supérieure} {\emph{Concentration technologie de l'information et télécommunications.}}
  \fi

  \ifenglish
    \entry {2001-2005} {B.ing. in electrical engineering} {École de technologie supérieure} {\emph{TODO}}
  \else
    \entry {2001-2005} {DEC en technologie de systèmes ordinés} {Cégep de l’Outaouais} {}
  \fi
\end{entrylist}


% \cventry{2008--2009}{Concepteur de matériel didactique}{École de technologie supérieure}{}{}{
% Responsable de l’élaboration des exercices et des outils nécessaires au bon déroulement des laboratoires du cours Systèmes répartis dans le domaine de la santé ((gts\footnotesize840})).
% \begin{itemize}
%   \item Assembler/développer des logiciels de type clients/serveurs suivant les spécifications et standards HL7 et DICOM pour les dossiers patients
%   \item Concevoir, créer et mettre au point les énoncés de laboratoire et les méthodes nécessaires à l’évaluation des étudiants
% \end{itemize}}

\ifenglish\section{experience}\else\section{expérience}\fi

\begin{entrylist}
  \ifenglish
    \entry {2009-2014} {chargé de cours} {École de technologie supérieure} {\emph{Environnement de développement de logiciels (ele{\footnotesize116})\\Analyse et traitement d'images (ele{\footnotesize747})}}
  \else
    \entry {2009-2014} {chargé de cours} {École de technologie supérieure} {\emph{Environnement de développement de logiciels (ele{\footnotesize116})\\Analyse et traitement d'images (ele{\footnotesize747})}}
  \fi

  \ifenglish
    \entry {2009-2014} {chargé de laboratoires} {École de technologie supérieure} {\emph{Systèmes répartis dans le domaine de la santé (gts{\footnotesize840})\\Environnement de développement de logiciels (ele{\footnotesize116})\\Analyse et traitement d'images (ele{\footnotesize747})}}
  \else
    \entry {2009-2014} {chargé de laboratoires} {École de technologie supérieure} {\emph{Systèmes répartis dans le domaine de la santé (gts{\footnotesize840})\\Environnement de développement de logiciels (ele{\footnotesize116})\\Analyse et traitement d'images (ele{\footnotesize747})}}
  \fi

  \ifenglish
    \entry {2009} {développeur logiciel} {CAE inc.} {\emph{Simulation d'affichage tête haute pour simulateur de vol militaire}}
  \else
    \entry {2009} {développeur logiciel} {CAE inc.} {\emph{Simulation d'affichage tête haute pour simulateur de vol militaire}}
  \fi

  \ifenglish
    \entry {2006-2008} {auxiliaire de recherche} {École de technologie supérieure} {\emph{Développement d'un outils de validation de conformité au standard hl{\small7}v{\small3}.\\Implémentation et évaluation d'un système de <<streaming>> d'image médicales}}
  \else
    \entry {2006-2008} {auxiliaire de recherche} {École de technologie supérieure} {\emph{Développement d'un outils de validation de conformité au standard hl{\small7}v{\small3}.\\Implémentation et évaluation d'un système de <<streaming>> d'image médicales}}
  \fi
\end{entrylist}

\ifenglish\section{awards}\else\section{récompenses}\fi

\begin{entrylist}
  \ifenglish
    \entry {2014} {best scientific poster at the annual SIIM meeting} {500\$} {\vspace{-3mm}}
  \else
    \entry {2014} {meilleur affiche scientifique à la rencontre annuelle SIIM 2014} {500\$} {\vspace{-3mm}}
  \fi
  \ifenglish
    \entry {2011} {bourse d’études supérieures Alexander-Graham-Bell du CRSNG} {35,000\$/an} {\vspace{-3mm}}
  \else
    \entry {2011} {bourse d’études supérieures Alexander-Graham-Bell du CRSNG} {35,000\$/an} {\vspace{-3mm}}
  \fi
  \ifenglish
    \entry {2011} {bourse d'excellence pour les diplômés de 1er cycle de l'ÉTS} {20,000\$/an} {\vspace{-3mm}}
  \else
    \entry {2011} {bourse d'excellence pour les diplômés de 1er cycle de l'ÉTS} {20,000\$/an} {\vspace{-3mm}}
  \fi
  \ifenglish
    \entry {2011} {bourse de doctorat en recherche du FQRNT} {20,000\$/an} {\vspace{-3mm}}
  \else
    \entry {2011} {bourse de doctorat en recherche du FQRNT} {20,000\$/an} {\vspace{-3mm}}
  \fi
  \ifenglish
    \entry {2009} {bourse de maîtrise en recherche du FQRNT} {15,000\$/an} {\vspace{-3mm}}
  \else
    \entry {2009} {bourse de maîtrise en recherche du FQRNT} {15,000\$/an} {\vspace{-3mm}}
  \fi
  \ifenglish
    \entry {2006} {bourse de maîtrise en recherche du FQRNT} {4,500\$} {\vspace{-3mm}}
  \else
    \entry {2006} {bourse de maîtrise en recherche du FQRNT} {4,500\$} {\vspace{-3mm}}
  \fi
\end{entrylist}

\ifenglish\section{implication bénévole}\else\section{implication bénévole}\fi
\begin{entrylist}
  \ifenglish
    \entry {depuis 2013} {administrateur d'un OBNL} {Association étudiante de l'ÉTS} {vice-président aux services\\représentant étudiant à la commission des études}
  \else
    \entry {depuis 2013} {administrateur d'un OBNL} {Association étudiante de l'ÉTS} {vice-président aux services\\représentant étudiant à la commission des études}
  \fi
\end{entrylist}
\newpage
% \section{bourses et récompenses}
% \cventry{2011}{\small bourse d’études supérieures Alexander-Graham-Bell du CRSNG}{35,000\$/an}{}{}{
% La prestigieuse bourse d’études supérieures du Canada (BESC) Alexander-Graham-Bell a été créée afin d’assurer un bassin fiable de personnes hautement qualifiées en vue de satisfaire aux besoins de l’économie du savoir du Canada. Les titulaires d’une BESC Alexander-Graham-Bell aideront à renouveler le corps professoral des universités canadiennes et seront les chefs de file de demain en recherche.}
% \cventry{2011}{\small bourse d’excellence pour les diplômés de 1er cycle de l’ÉTS}{20,000\$/an}{}{}{}
% \cventry{2011}{\small bourse de doctorat en recherche du FQRNT}{20,000\$/an}{}{}{}
% \cventry{2009}{\small bourse de maîtrise en recherche du FQRNT}{15,000\$/an}{}{}{}
% \cventry{2006}{\small bourse de recherche de 1er cycle du CRSNG}{4,500\$}{}{}{}
% \section{publications}

\ifenglish\section{publication}\else\section{publication}\fi
\printbibsection{article}{\ifenglish articles in peer-reviewed journals\else articles de journaux révisés par les pairs\fi}
\begin{refsection}
  \nocite{*}
  \printbibliography[sorting=chronological, type=inproceedings, title={\ifenglish international peer-reviewed conferences/proceedings \else articles de conférence internationale révisés par les pairs \fi}, notkeyword={france}, heading=subbibliography]
\end{refsection}
% \begin{refsection}
  % \nocite{*}
  % \printbibliography[sorting=chronological, type=inproceedings, title={local peer-reviewed conferences/proceedings}, keyword={france}, heading=subbibliography]
% \end{refsection}
% \printbibsection{misc}{other publications}
%\printbibsection{report}{rapport technique}

\end{document}
